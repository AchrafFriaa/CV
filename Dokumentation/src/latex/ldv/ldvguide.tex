% !TEX program = lualatex
% !TEX spellcheck = de-DE
\ProvidesFile{ldvguide.tex}[2019/03/18 2.3 ldv]

\documentclass[index=totoc]{scrdoc}
\usepackage[ngerman]{babel}
\usepackage[utf8]{inputenc}
\usepackage{amsmath,amssymb}
\usepackage{metalogo}
%\usepackage{etoolbox}
\usepackage{expl3}
\ExplSyntaxOn{}
\sys_if_engine_luatex:TF
{
  \usepackage{fontspec}
  \defaultfontfeatures{Scale=MatchLowercase}
  \defaultfontfeatures[TUM Neue Helvetica]{
    UprightFont = TUM Neue Helvetica 55 Regular,
    BoldFont = TUM Neue Helvetica 75 Bold,
    ItalicFont = TUM Neue Helvetica 56 Italic,
    BoldItalicFont = TUM Neue Helvetica 76 Bold Italic,
    Ligatures = TeX,
    Scale = 0.92,
  }
  \IfFontExistsTF{Times}{\setmainfont{Times}}{}
  \IfFontExistsTF{Times New Roman}{\setmainfont{Times New Roman}}{}
  \IfFontExistsTF{Arial}{\setsansfont{Arial}}{}
  \IfFontExistsTF{Helvetica}{\setsansfont{Helvetica}}{}
  \IfFontExistsTF{TUM Neue Helvetica}{\setsansfont{TUM Neue Helvetica}}{}
  \IfFontExistsTF{DejaVu Sans Mono}{\setmonofont{DejaVu Sans Mono}}{}
}
{
    \sys_if_engine_xetex:TF
    {
      \usepackage{fontspec}
      \defaultfontfeatures{Scale=MatchLowercase}
      \defaultfontfeatures[TUM Neue Helvetica]{
        UprightFont = TUM Neue Helvetica 55 Regular,
        BoldFont = TUM Neue Helvetica 75 Bold,
        ItalicFont = TUM Neue Helvetica 56 Italic,
        BoldItalicFont = TUM Neue Helvetica 76 Bold Italic,
        Ligatures = TeX,
        Scale = 0.92,
      }
      \IfFontExistsTF{Times}{\setmainfont{Times}}{}
      \IfFontExistsTF{Times New Roman}{\setmainfont{Times New Roman}}{}
      \IfFontExistsTF{Arial}{\setsansfont{Arial}}{}
      \IfFontExistsTF{Helvetica}{\setsansfont{Helvetica}}{}
      \IfFontExistsTF{TUM Neue Helvetica}{\setsansfont{TUM Neue Helvetica}}{}
      \IfFontExistsTF{DejaVu Sans Mono}{\setmonofont{DejaVu Sans Mono}}{}
    }
    {
        \renewcommand{\rmdefault}{ptm}
        \usepackage[scaled=0.92]{helvet}
    }
}
\ExplSyntaxOff{}
\renewcommand\familydefault{\sfdefault}


\usepackage[autostyle=tryonce]{csquotes}
\usepackage[unicode]{hyperref}
\hypersetup{
				breaklinks,
				colorlinks,
				pdfauthor={Walter Bamberger, Martin Knopp},
				pdfkeywords={ldvklassen, ldv, latex},
				pdfpagelayout=OneColumn,
				pdfstartview=FitH,
				pdfsubject={Dokumentation LaTeX Klassen},
				pdftitle={Die LDV-Klassen},
				anchorcolor={black},
				citecolor={black},
				filecolor={black},
				linkcolor={black},
				menucolor={black},
				urlcolor={black},
				}

\tolerance=1500
%\EnableCrossrefs
\CodelineIndex
\RecordChanges


\GetFileInfo{ldvguide.tex}
\title{Die LDV-Klassen%
\footnote{Dies ist Version \fileversion.}}
\date{\filedate}
\author{Walter Bamberger, Martin Knopp}

\begin{document}

\maketitle
\tableofcontents

\section{Einführung}%
\label{sec:einfuehrung}


Die Installation dieser \LaTeX-Klassen erklären die Dokumente
\enquote{Installationsanleitung.txt} und \enquote{Installation instructions.txt}.

Das LDV-Paket bietet zwei Dokumentenklassen an: \emph{ldvarticle} ist
für kürzere Dokumente (üblicherweise zwischen 1 und 25 Seiten)
gedacht. Es verwendet hierfür einseitigen Druck und beginnt die
Überschriftenhierarchie mit der
\verb|\section|-Ebene. \emph{ldvbook} zielt dagegen auf größere
Dokumente (ab etwa 15 Seiten) ab. Es stellt dazu doppelseitigen Druck
ein, bietet die Überschriftenebene \verb|\chapter| und beginnt jedes
Hauptkapitel auf einer neuen ungeraden Seite. Mit diesen
Dokumentenklassen können alle anvisierten wissenschaftlichen Dokumente
(siehe Kapitel~\ref{sec:dokumentarten}) schnell und einfach umgesetzt
werden. Der Fokus der Erweiterungen und Verbesserungen liegt vor allem
auf der Titelei und den Literaturverweisen.

Wie die Namen der Dokumentenklassen bereits nahelegen, sind diese
verwandt mit den entsprechenden Standardklassen bzw. den
entsprechenden \KOMAScript-Klas\-sen. Warum wurden dann neue Klassen ins
Leben gerufen und welchen Vorteil bringen sie Ihnen?

\begin{description}
\item[Corporate Identity.]  Die LDV-Klassen setzten, wo sinnvoll und
  möglich, die Intentionen und Vorgaben des neuen Style Guide der TUM
  um. Dies reicht von der Schriftenauswahl, über die Definition der
  Farben bis zur Gestaltung der Titelseite.
\item[Einstiegskomplexität.]  Die LDV-Klassen sollen \LaTeX-Neulingen
  (also den meisten Studierenden) den Einstieg möglichst einfach
  machen. Dazu ermöglichen sie eine sehr einfache \LaTeX-Präambel,
  setzen eine ausgewogene und moderne Layoutvorgabe um und bieten vor
  allem im Bereich der Titelei einige Automatismen.
\item[Metadatenverarbeitung.]  Die Standardklassen von \LaTeX\ nutzen
  die Metainformationen wie Autor und Titel lediglich zur Gestaltung
  der Titelseite. Die LDV-Klassen verwenden diese Informationen
  dagegen an möglichst vielen weiteren Stellen: Auf der Umschlagseite,
  auf der Impressumsseite und insbesondere auch in den
  PDF-Dokumenteneigenschaften. Sie bekommen also ohne weiteres Zutun
  ein komplettes Dokumentengerüst, einschließlich der
  PDF-Meta\-daten.
\item[Umschlagseite.]  Umfangreiche Werke (Bücher) sind
  zusätzlich zur Titelseite (zumeist die Seite~5) von einem Umschlag
  umgeben, der Raum zur individuellen Gestaltung bietet. Im
  Gegensatz zu den Standardklassen von \LaTeX, integrieren die
  LDV-Klassen Funktionalitäten zum einfachen Umgang mit dem Umschlag.
\item[Titelei.]  Das Makro \verb|\maketitle| der LDV-Klassen besitzt
  erweiterte Mög\-lich\-kei\-ten, um automatisch eine komplette
  Titelei zu generieren. Es erzeugt insbesondere eine Impressumsseite,
  auf der optional Lizenzinformationen stehen. Die sechs Creative
  Commons-Lizenzen (CC-BY, CC-BY-SA, CC-BY-ND, CC-BY-NC, CC-BY-NC-SA,
  CC-BY-NC-ND) sind bereits in die Klassen integriert und können mit
  dem Befehle \verb|\license| ausgewählt werden (siehe
  http://creativecommons.org).
\item[Literaturverzeichnis.]  Die LDV-Klassen beherrschen das
  Autor-Jahr-Sche\-ma, mit dem man den Literaturverweis gut in den
  Text integrieren kann und ein Werk bereits im Text gut
  wiedererkennen kann. Darüber hinaus beinhalten sie einen eigenen
  Literaturverzeichnisstil, der mit den modernen Attributen DOI, ISBN,
  ISSN und URL umgehen kann, und der den Dokumententyp \enquote{www} für
  Webseiten und \enquote{media} für Mediendateien kennt. Bei diesen Quellen
  gibt es häufig eine große Unsicherheit im Umgang.
\item[Zweisprachige Umsetzung.]  Alle Funktionen, die Text im Dokument
  erzeugen, sind konsequent zweisprachig aufgebaut, für deutsche
  (\verb|lang=| \verb|ngerman|) und für englische
  (\verb|lang=englisch|) Texte, indem sie sich in das Rahmenkonzept
  des Babel-Pakets integrieren. Die Standardsprache ist Englisch.
\end{description}

Das folgende Beispiel zeigt das Grundgerüst für eine Diplomarbeit.

\begin{verbatim}
\documentclass[doctype=Diplomarbeit,lang=ngerman]{ldvbook}

\begin{document}

\title{Der große Wurf}
\author{H. Mustermann}
\license{CC-BY}
\supervisor{W. Bamberger}

\maketitle[frontcover=Design1]
\tableofcontents

\chapter{Einführung}

...

\bibliography{diplomarbeit}

\end{document}
\end{verbatim}
%
Man sieht die sehr kurze \LaTeX-Präambel. Und auch die Titelei ist mit
wenigen Zeilen getan. Für eine normale Diplomarbeit dürfte das Gerüst
genügen; denn folgende Pakete sind so bereits automatisch eingebunden:
%
\begin{itemize}
\item inputenc
\item fontenc
\item babel
\item array
\item fancyvrb
\item color
\item graphicx
\item amsmath
\item amssymb
\item natbib
\item hyperref
\item varioref
\item helvet (je nach Klassenoption)
\end{itemize}

Indem die LDV-Klassen das komplette \LaTeX-System sinnvoll
vorkonfigurieren einschließlich aller üblichen Pakete, erlauben sie
Neulingen einen sehr schnellen Einstieg.

Diese Anleitung ist keine \LaTeX-Anleitung. Vielmehr beschreibt sie
nur die Zusätze, die die LDV-Klassen im Vergleich zu den
\KOMAScript-Klassen bieten. Ich verweise jedoch immer wieder auf die
Beschreibungen der zu einem Thema wichtigen Pakete.




\section{Anvisierte Arten von Dokumenten}%
\label{sec:dokumentarten}

Die LDV-Dokumentenklassen zielen auf strukturierte Dokumente mit meist
wissenschaftlichem Hintergrund ab. Bei der Entwicklung habe ich vor
allem an
%
\begin{itemize}
\item Vorlesungsskripte,
\item Studentische Abschlussarbeiten (Studien-, Diplom-, Bachelor- und
  Masterarbeiten sowie interdisziplinäre Projekte),
\item Doktorarbeiten,
\item Forschungsberichte und
\item wissenschaftliche Aufsätze/Paper
\end{itemize}
%
gedacht. Sie bestehen im Wesentlichen aus der Titelei, Verzeichnissen
und Fließtext mit Tabellen und Abbildungen. Im Vergleich zu den
\KOMAScript-Klassen benötigt man für diese Dokumente vor allem
Ergänzungen im Bereich der Titelei und dem
Literaturverzeichnis. Hierin lag deshalb das Augenmerk für die
Entwicklung der LDV-Klassen.

Der volle Funktionsumfang der LDV-Klassen steht in den Sprachen
Deutsch und Englisch zur Verfügung. Eine Erweiterung um weitere
Sprachen ist denkbar und einfach möglich, aber im Augenblick nicht
geplant.

Briefe sowie Texte mit freiem Layout decken diese Dokumentenklassen
nicht ab. Dagegen ist eine Erweiterung in Richtung eines
Konferenzbands (proceedings) denkbar.




\section{Sprache und Kodierung}%
\label{sec:kodi-und-sprache}

\DescribeOption{lang}%
Die LDV-Klassen laden automatisch das \emph{Babel}-Paket mit den
Einstellungen zur englischen Sprache. Zusätzliche Spracheinstellungen
können Sie mit der Klassenoption \verb|lang| laden, also
beispielsweise mit der Option \verb|lang=ngerman|. (Bitte bevorzugen
Sie \verb|ngerman| gegenüber \verb|german|. Letzteres ist veraltet.)

\DescribeOption{inputenc}%
Im Sinne einer modernen Sprachunterstützung wählen die LDV-Klassen als
Zeichensatz automatisch UTF-8, indem sie das Paket \emph{inputenc} mit
der entsprechenden Option laden. Wollen Sie eine andere
Zeichenkodierung für Ihre .tex-Datei verwenden, dann müssen Sie die
Klassenoption \verb|inputenc| verwenden. In welchem Format Ihre
.tex-Datei kodiert ist, bestimmt Ihr \TeX-Editor.

Die LDV-Klassen verwenden automatisch die T1-kodierten Schriften von
\LaTeX, die modernere und flexiblere Kodierung. Dazu laden sie das
Paket fontenc mit Option \enquote{T1}. Dieses Verhalten ist fest vorgegeben
und unveränderlich.

\paragraph{Beispiel}

\begin{verbatim}
\documentclass[lang=ngerman,inputenc=latin1]{ldvbook}
\end{verbatim}




\section{Meta-Informationen}%
\label{sec:meta}

\DescribeMacro{\author}%
\DescribeMacro{\citationaddress}%
\DescribeMacro{\institute}%
\DescribeMacro{\keywords}%
\DescribeMacro{\license}%
\DescribeMacro{\licensetext}%
Die LDV-Klassen können bestimmte bibliographische Informationen an
verschiedenen Stellen in einem Dokument einfügen: Auf der
Umschlagseite, auf der Titelseite, auf der Impressumsseite (siehe
Kapitel~\ref{sec:titelei}) und in den Dokumenteneigenschaften der
PDF-Datei (nur mit pdf\LaTeX). Dabei beachten sie die gewählte
Textsprache (deutsch und englisch). Im Einzelnen sind das folgende
Meta-Informationen:

\DescribeMacro{\postaddress}%
\DescribeMacro{\publishers}%
\DescribeMacro{\publishersurl}%
\DescribeMacro{\subtitle}%
\DescribeMacro{\subject}%
\DescribeMacro{\title}%
\DescribeMacro{\version}%
\begin{itemize}
\item der Verfasser (\verb|\author|),
\item der Titel (\verb|\title|),
\item der Untertitel (\verb|\subtitle|),
\item die Dokumentenart bzw. das Thema (\verb|\subject|, manchmal auch
  als Betreff beschrieben),
\item die veröffentlichende Einrichtung bzw. Person
  (\verb|\publishers|, z.B. die Universität),
\item die Internetadresse der Einrichtung (\verb|\publishersurl|),
\item die Postadresse der Einrichtung (\verb|\postaddress|),
\item der Ort der Einrichtung, wie er in der Referenzierung erscheinen
  soll\\ (\verb|\citationaddress|),
\item der Lehrstuhl (\verb|\institute|),
\item die Versionsnummer (\verb|\version|),
\item die Schlagwörter (\verb|\keywords|) und
\item die Lizenz (den kompletten Lizenztext mit \verb|\licensetext|
  oder eine der vordefinierten Lizenzen mit \verb|\license|).
\end{itemize}
%
Bei studentischen Abschlussarbeiten kommt der Betreuer
(\verb|\supervisor|) hinzu.

Die Werte für \verb|\citationaddress|, \verb|\institute|,
\verb|\postaddress|, \verb|\publishers|, und \verb|\publishersurl|
sind bereits mit den passenden Werten für unseren Lehrstuhl vorbelegt.
Ein Lizenztext kann sehr einfach mit \verb|\license| ausgewählt
werden.

\DescribeMacro{\keywordsname}%
Die Schlagwörter werden mit einem entsprechenden Wort eingeleitet
(z.B. Schlagwörter oder Key words).  Dieses Wort ist in
\verb|keywordsname| lokalisiert gespeichert.



\paragraph{Lizenzen.}

Texte und Bilder (nicht deren Inhalt) sind nach dem Urheberrecht
geschützt. Sie dürfen von anderen nicht ohne Erlaubnis benutzt
werden. Will ein Anderer beispielsweise ein Bild verwenden, muss er
individuell um Erlaubnis fragen, also eine Lizenz erwerben. Der Autor
selbst kann dies aber vereinfachen, indem er das Werk unter eine
Lizenz für die generelle Öffentlichkeit stellt. Mehr zu den Gründen,
warum das gut sein kann und wie das geht, beschreibt die Website von
Creative Commons (creativecommons.org).

Die Organisation \emph{Creative Commons} hat dazu ein modulares
Lizenzsystem entwickelt. Die LDV-Dokumentenklassen bieten einen
vereinfachten Zugriff auf diese sechs Lizenzen. Wählen Sie mit dem
Makro \verb|\license| eine der Lizenzen aus. Die Lizenzen werden über
ihr Kürzel angegeben: CC-BY, CC-BY-SA, CC-BY-ND, CC-BY-NC,
CC-BY-NC-SA, CC-BY-NC-ND. Was sich hinter diesen Zeichen verbirgt,
finden Sie ausführlich erläutert auf der Website von Creative Commons
unter \url{www.creativecommons.org}.

Wer eine andere als diese sechs Lizenzen verwenden will, kann den
Lizenztext mit dem Makro \verb|\licensetext| eingeben.


\paragraph{Beispiel.}

Das Skript zum Praktikum Informatik definiert seine bibliographischen
Informationen mit folgendem Code:

\begin{verbatim}
\title{Programmieren in C}
\subtitle{Der C-Kurs zum Praktikum Informatik}
\author{K. Centmayer\and F. Obermeier}
\version{1.1}
\subject{Praktikumsskript}
\keywords{C, Programmieren, Programmierkurs}
\end{verbatim}


\paragraph{Weitere Dokumentation}%
\label{sec:weit-dokum}

Die meisten Makros zum Steuern des Titelinhalts entstammen den
\KOMAScript-Klas\-sen und sind somit in scrguide.pdf (deutsch)
bzw. scrguien.pdf (englisch) dokumentiert.




\section{Titelei und Umschlag}%
\label{sec:titelei}

Die Titelei bezeichnet den Teil eines Buches, der dem Textteil
vorausgeht. Häufig besitzen dieser Bereich Seitenzahlen aus römischen
Ziffern. (Vergleiche dazu den Artikel \enquote{Titelei} in \enquote{Wikipedia, Die
freie Enzyklopädie}.) Die Titelei besteht aus
%
\begin{itemize}
\item der Schmutztitelseite oder dem Vortitel (Seite 1),
\item der Frontispizseite (Seite 2, eine Illustration, heute häufig
  unbedruckt),
\item dem Titelblatt bzw. der Titelseite (Seite 3)
\item der Impressumsseite (Seite 4),
\item einer Widmungsseite (Seite 5),
\item den Vorworten sowie
\item dem Inhaltsverzeichnis und anderen Verzeichnissen.
\end{itemize}

\DescribeMacro{\maketitle}%
Einige Teile davon sind optional; erwähnenswert ist darüber hinaus,
dass der Umschlag eines Buches nicht zur Titelei zählt. Die
LDV-Klassen können mit dem Befehl \verb|\maketitle| die ersten fünf
der oben genannten Punkte setzen sowie die Umschlagseite. Die Vorworte
können Sie als Kapitel mit den Sternvarianten der Gliederungsbefehle
kodieren. Und die diversen Verzeichnisse können Sie mit den üblichen
\LaTeX-Werkzeugen erzeugen.

Um die neuen Funktionen bezüglich des Umschlags flexibel in das Makro
\verb|\maketitle| integrieren zu können, verwendet dieses Makro in den
Optionen Schlüssel-Wert-Paare. Der unten dargestellte Code weist
beispielsweise dem Schlüssel \verb|frontcover| den Wert \verb|Design1|
zu.

\paragraph{Einführendes Beispiel.}

\DescribeMacro{\maketitle}%
\DescribeMacro{\extratitle}%
\DescribeMacro{\dedication}%
Der folgende \LaTeX-Code realisiert ein sehr umfangreiches Beispiel,
welches all die genannten Fähigkeiten zeigt. Er erzeugt ein
elfseitiges Dokument, acht davon generiert der Aufruf von
\verb|\maketitle|: Die Seiten -1 und 0 entfallen dabei auf den
Umschlag, Seite 1 beinhaltet den Schmutztitel, Seite 2 ist leer, Seite
3 zeigt den Haupttitel (Titelseite), Seite 4 die Impressumsseite,
Seite 5 die Widmung, Seite 6 bleibt leer, Seite 7 beginnt das Vorwort
und Seite 11 das Inhaltsverzeichnis. Bei kleineren Werken entfallen
häufig der Schmutztitel, die Widmung und das Vorwort.

\begin{verbatim}
\documentclass[lang=ngerman]{ldvbook}
\begin{document}

\title{Programmieren in C}
\subtitle{Der C-Kurs zum Praktikum Informatik}
\author{K. Centmayer\and F. Obermeier}
\version{1.1}
\subject{Praktikumsskript}
\keywords{C, Programmieren, Programmierkurs}

\extratitle{{\bfseries\Large Programmieren in C}}     % Schmutztitel (optional)
\dedication{Gewidmet meiner Frau Theresa und meinem Freund Johannes,\\
  für ihre Geduld und ihre tatkräftige Unterstützung} % Widmung (optional)

\maketitle[frontcover=Design1]

\chapter*{Vorwort}  % Optional
Ich würde mal sagen, Text, Text, Text.

\tableofcontents
\end{document}
\end{verbatim}
%
Zu den Makros \verb|\extratitle|, \verb|\dedication| und
\verb|\subtitle| erfahren Sie mehr in der Dokumentation zu den
\KOMAScript-Klassen. Kapitel~\ref{sec:meta} bespricht die
Makros \verb|\version| und \verb|\keywords|.

Zusammenfassend haben Schmutztitelseite, Widmung und Vorwort bei den
von den LDV-Klassen anvisierten Dokumentenarten eine geringe
Bedeutung. Die Trennung von einem visuell orientierten Umschlag und
der Titelseite ist aus gestalterischer Sicht dagegen relevant. Deshalb
befasst sich das verbleibende Kapitel mit der Umschlagseite, der
Titelseite und der Impressumsseite. Zusätzlich bieten die LDV-Klassen
in Bezug auf die Titelei auch noch einige Zusatzfunktionen für
studentische Abschlussarbeiten, die im Abschluss vorgestellt werden.

\paragraph{Seitennummerierung.}

\DescribeOption{pagenumber}%
Über die Option \verb|pagenumber| können Sie die Seitennummer der
ersten Seite der Titelei festlegen. Standardmäßig ist dies die Seite
1. Dies kann nützlich sein, falls vorausgehende Seiten außerhalb des
\LaTeX-Dokuments erstellt werden.

\paragraph{Beispiel.} Das folgende Beispiel erzeugt keinen Umschlag
und keinen Schmutztitel. Die Titelseite ist die erste von \LaTeX\
generierte Seite und erhält die Seitennummer 3.

\begin{verbatim}
\title{Programmieren in C}
\subtitle{Der C-Kurs zum Praktikum Informatik}
\author{K. Centmayer\and F. Obermeier}

\maketitle[pagenumber=3]
\end{verbatim}

\paragraph{Weitere Dokumentation.}

Viele Makros rund um die Titelei sind in scrguide.pdf (deu\-tsch)
bzw.\ scrguien.pdf (englisch) dokumentiert.




\subsection{Umschlag}%
\label{sec:umschlag}

Der Umschlag, auch als Buchdeckel oder cover bezeichnet, bietet Raum
für eine individuelle Gestaltung, frei vom Seitenspiegel und den
anderen Layoutvorgaben des Buches. Er ist kein Teil der Titelei und
geht auch nicht in die Seitenzählung ein. Um der eigenen Arbeit trotz
der sonstigen Layoutvorgaben in \LaTeX\ einen individuellen Charakter
verleihen zu können, bieten die LDV-Klassen die Möglichkeit, eines von
mehreren Coverdesigns auszuwählen oder eine Umschlagseite einzubinden,
die Sie in einem anderen Layoutprogramm erstellt haben.

\DescribeOption{frontcover}%
Zur Zeit gibt es nur ein vorgegebenes Coverdesign. Es wird mit der
Option \verb|frontcover =Design1| des Makros \verb|\maketitle|
aktiviert.

\paragraph{Beispiel.} Das folgende Beispiel erzeugt eine
Umschlagseite, eine leere Umschlagrückseite, eine Titelseite und eine
Impressumsseite.

\begin{verbatim}
\title{Programmieren in C}
\subtitle{Der C-Kurs zum Praktikum Informatik}
\author{K. Centmayer\and F. Obermeier}

\maketitle[frontcover=Design1]
\end{verbatim}

\paragraph{ToDo}

\begin{itemize}
\item Alternativ kann auch eine selbst gestaltete Umschlagseite
  (PDF mit einer Seite) eingebunden werden, mit der Option
  \verb|coverfile|. Beispiel:
\begin{verbatim}
\maketitle[frontcoverfile=meinumschlag.pdf]
\end{verbatim}
\end{itemize}



\subsection{Titelseite}%
\label{sec:titelseite}

Die Titelseite beinhaltet die wichtigsten bibliographischen
Informationen auf einer Seite; sie erscheint bei Büchern zumeist auf
Seite 3. Die LDV-Klassen ordnen auf dieser Seite
%
\begin{itemize}
\item die Dokumentenart (\verb|\subject|),
\item den Titel (\verb|\title|),
\item den Untertitel (\verb|\subtitle|),
\item den Autor bzw. die Autoren (\verb|\author|),
\item das Erscheinungsdatum bzw. Kompilierdatum (\verb|\date|),
\item eine Versionsinformation (\verb|\version|, ähnlich einer
  Auflagennummer),
\item einen freien Titelkopf (\verb|\titlehead|) sowie
\item das TUM- und LDV-Logo mit den Namen von Universität und
  Lehrstuhl
\end{itemize}
%
an. Einige dieser Informationen sind optional.
Der Wert von \verb|\publishers| findet im Gegensatz zu den
Standardklassen keine Beachtung, weil die Institutionsnamen und Logos
bereits fest vorgegeben sind.

\DescribeOption{titlepage}%
Die Klassenoption \verb|titlepage=true| erzeugt einen ganzseitigen
Titel wie oben beschrieben. Dies ist die Vorgabe bei der
Dokumentenklasse ldvbook. Die Klassenoption \verb|titlepage=false|
setzt den Titel dagegen an den Seitenkopf; darunter beginnt dann
gleich der normale Text. Dies ist die Standardeinstellung für die
Dokumentenklasse ldvarticle. Bei diesem verkürzten Titel am Seitenkopf
werden die oben genannten Elemente anders angeordnet und keine
Institutionsnamen abgedruckt; die Logos bleiben erhalten.


\subsection{Impressumsseite}%
\label{sec:impressumsseite}

Auf der Rückseite der Titelseite folgt die Impressumsseite, zumeist
also auf der Seite 4. Sie enthält detaillierte Informationen zum Werk,
vor allem Daten, die das Urheberrecht und Bibliotheken fordern. Die
Informationen auf der Impressumsseite sind demgemäß ausschlaggebend
für Zitate.

Die LDV-Klassen erzeugen die Impressumsseite nur bei doppelseitigem
Druck. Sie beinhaltet dann
%
\begin{itemize}
\item ein Zitierungsbeispiel bestehend aus dem Autor (\verb|\author|),
  dem Titel (\verb|\title|) mit dem Untertitel (\verb|\subtitle|), der
  Version (\verb|\version|), der Dokumentenart (\verb|\subject|), der
  veröffentlichende Institution (\verb|\publishers|), dem Ort der
  Institution (\verb|\citationaddress|) und dem Erscheinungsjahr
  (\verb|\year|),
\item die charakterisierenden Schlüsselwörter (\verb|\keywords|),
\item die Urheberangabe bestehend aus dem Jahr (\verb|\year|) und den
  Autoren (\verb|\author|),
\item die Kontaktdaten bestehend aus dem Lehrstuhlnamen
  (\verb|\institute|), dem Universitätsnamen (\verb|\publishers|), der
  Postadresse (\verb|\postaddress|) und der Internet-Adresse
  (\verb|\publishersurl|),
\item eine Lizenz (\verb|\license| oder \verb|\licensetext|) sowie
\item freie Zusatzinformationen oben auf der Seite (z.B. zur
  Umschlagseite, mittels \verb|\uppertitleback|).
\end{itemize}

Bei studentischen Abschlussarbeiten erscheinen hier auch noch
Informationen zu den Betreuern (siehe
Abschnitt~\ref{sec:sond-fur-stud}). Viele der oben genannten Punkte
sind optional und erscheinen deshalb nur, falls sie angegeben wurden.


\subsection{Besondere Funktionen für studentische Abschlussarbeiten}%
\label{sec:sond-fur-stud}

Für studentische Abschlussarbeiten beinhalten die LDV-Klassen einige
Automatismen. Sie sollen den Unsicherheiten bei Studenten
entgegenwirken, welche Informationen denn wo in der Arbeit erscheinen
sollen.

\DescribeOption{doctype}%
Dazu geben Sie zuerst je nach Typ der Arbeit eine der folgenden
Klassenoptionen an:
%
\begin{itemize}
\item \verb|doctype=mastersthesis|
\item \verb|doctype=bachelorsthesis|
\item \verb|doctype=Diplomarbeit|
\item \verb|doctype=Studienarbeit|
\item \verb|doctype=IDP|
\end{itemize}
%
Damit werden die Funktionen und Einstellungen für studentische
Abschlussarbeiten aktiviert.

\DescribeMacro{\supervisor}%
Neben den Titel und dem Autor müssen Sie dann noch den Betreuer
angegeben. Die LDV-Klassen bieten dazu das Makro
\verb|\supervisor|. Abschließend stellt ein Aufruf von
\verb|\maketitle| alle relevanten Informationen zusammen.

\paragraph{Beispiel.}

Den Anfang einer Masterarbeit zeigt folgendes Beispiel. Es erzeugt ein
vierseitiges Dokument mit allen prüfungsrelevanten
Rahmeninformationen.

\begin{verbatim}
\documentclass[doctype=mastersthesis]{ldvbook}

\begin{document}

\title{Modeling a machine-to-machine relaying scenario with ad-hoc segments}
\author{Chunlong Tang}
\supervisor{W. Bamberger}

\maketitle[frontcover=Design1]

...

\end{document}
\end{verbatim}

\paragraph{Hinweise.}

Den LDV-Klassen liegt ein etwas umfassenderes Grundgerüst einer
Diplomarbeit bei (diplomarbeit.tex). Bitte benutzen Sie dieses als
Vorlage für ihre Abschlussarbeit. Darüber hinaus gibt es auch ein
umfangreiches Dokument mit Tipps zur Ausarbeitung. Bitte lesen Sie
dieses zu Anfang aufmerksam durch.




\section{Textauszeichnung}%
\label{sec:textauszeichnung}

Die Fähigkeiten von \LaTeX\ und den \KOMAScript-Klassen für Fließtext
sind sehr umfangreich und zumeist ausreichend. Die LDV-Klassen
erweitern sie in diesem Bereich nur um wenige Funktionen.


\subsection{Starke Hervorhebung}%
\label{sec:starke-hervorhebung}

\DescribeMacro{\emphemph}%
Zur Hervorhebung von Text bietet \LaTeX\ den Befehl \verb|\emph|. So
gekennzeichneter Text soll während des Lesens den Lesefluss
verändern. Zusätzlich sollen in manchen Texten gewisse Begriffe
bereits beim überfliegen des Textes auffallen, um Orientierung zu
bieten, ähnlich zu Überschriften. Dies ist eine stärkere
Hervorhebung. Dazu definieren die LDV-Klassen den Befehl
\verb|\emphemph|.

\paragraph{Beispiel}

\begin{verbatim}
Schließlich bildet der Bereich der \emphemph{Serviceroboter} ein
sehr vielversprechendes Anwendungsfeld.
\end{verbatim}


\subsection{Code in Überschriften und Bildunterschriften}%
\label{sec:code-uberschr-und}

\DescribeMacro{\simpleverb}%
\LaTeX\ bietet das Makro
\verb|\verb|, um vorformatierten Text, also beispielsweise Quellcode,
darzustellen. Optisch benutzt es dazu in der Regel die
Festweitenschrift Computer Modern Typewriter ??. Dieses Makro
funktioniert aber nicht innerhalb von Überschriften,
Bildunterschriften, usw. Um auch innerhalb solcher Makros Quellcode
einbetten zu können, bieten die LDV-Klassen das Makro
\verb|\simpleverb|.

Dieses neue Makro stellt Text genauso dar wie
\verb|\verb|, jedoch kann es nicht beliebige Zeichen unverändert
darstellen. Vielmehr müssen Sie als Autor die Steuerzeichen von
\LaTeX\ benutzen, um gewisse Sonderzeichen setzen zu
können. \verb|\simpleverb|
verhält sich wie ein normales \LaTeX-Makro. Genau deshalb kann es auch
innerhalb von Überschriften benutzt werden. Es setzt lediglich den
Inhalt in der passenden Darstellungsform -- in derselben wie
\verb|\verb|.

\paragraph{Beispiele}

\begin{verbatim}
\section{Ausgabe mit \simpleverb{printf}}
\end{verbatim}

\begin{verbatim}
\paragraph*{Was ergibt die logische Verknüpfung \simpleverb{c = a \&\& b}?}
\end{verbatim}


\subsection{Bemerkungen des Autors}%
\label{sec:bemerk-des-autors}

\DescribeEnv{note}%
In Büchern sieht man immer wieder am Ende von Abschnitten einen
abgesetzten Text mit Bemerkungen und Hinweisen des Autors. Dieses
Vorgehen soll zusätzliche Hinweise und Interpretationen vom
eigentlichen Inhaltsverlauf trennen. Die LDV-Klassen bieten hierfür
die Umgebung \verb|note|.

\paragraph{Beispiel}

\begin{verbatim}
\begin{note}
  Wann immer Sie in diesem Skript auf das Symbol links stoßen,
  finden Sie einen Hinweis, dass Sie im Quellcodeverzeichnis ein
  übersetzbares Beispiel zum behandelten Stoff finden. Alternativ
  wird Ihnen das entsprechende Programm direkt in einem Bild
  präsentiert (s.u.).
\end{note}
\end{verbatim}


\subsection{Abbildungen}%
\label{sec:abbildungen}

Die LDV-Klassen binden das \LaTeX-Paket \emph{graphicx} automatisch
mit ein, um den grundlegenden Umgang mit Bilddateien zu
ermöglichen. Mit pdflatex können Sie damit die Dateiformate PDF, PNG
und JPEG direkt in \LaTeX-Dokumente einbinden.

\DescribeMacro{\graphicswidth}%
\DescribeMacro{\graphicswidthtwo}%
Um ein stringentes Erscheinungsbild zu erhalten, druckt man die
Grafiken in einem Dokument in einheitlichen Breiten. Dazu definieren
die LDV-Klassen die Längenmaße \verb|\graphicswidth| und
\verb|\graphicswidthtwo|. Das erste ist die Breite eines Bildes mit
nahezu der Textbreite (abzüglich 2\,em für den Rand). Das zweite ist
die Breite eines Bildes, wenn zwei Bilder nebeneinander mit einem
Zwischenraum von 1\,em gedruckt werden sollen.

Natürlich passt die (große) Breite \verb|\graphicswidth| nicht zu
allen Bildern. Schmälere Bilder kann man dann von Text umflossen
einbetten. Dazu gibt es diverse \LaTeX-Pakete.

\paragraph{Beispiele}

\begin{verbatim}
\begin{figure}[htb]
  \centering%
  \includegraphics[width=\graphicswidthtwo]{img/7-1}%
  \caption{Bildunterschrift für das erste Bild.}
  \label{fig:bsp1}
\end{figure}
\end{verbatim}

\begin{verbatim}
\begin{figure}[htb]
  \centering%
  \includegraphics[width=\graphicswidthtwo]{img/7-2}%
  \quad%
  \includegraphics[width=\graphicswidthtwo]{img/7-3}%
  \caption{Bildunterschrift für die zweite Abbildung. Es kann auch
    mehr Text sein.}
  \label{fig:bsp2}
\end{figure}
\end{verbatim}




\section{Mathematik}%
\label{sec:mathematik}

Für den Satz von mathematischen Formeln binden die LDV-Klassen
automatisch die Pakete \emph{amsmath} und \emph{amssymb}
eingebunden.

\DescribeEnv{definition}%
\DescribeEnv{theorem}%
Auf deren Basis definieren die LDV-Klassen dann die beiden Umgebungen
\verb|definition| und \verb|theorem|. Ersteres ist ein nummerierter
Block für mathematische Definitionen. Zweiteres ein nummerierter Block
für mathematische Sätze. Die Bezeichnungen im Text stehen in den
Sprachen Deutsch und Englisch zur Verfügung.

\paragraph{Beispiele}

\begin{verbatim}
\begin{definition}
  \label{th:ueberzeugungsstaerkeverteilung}
  Gegeben sei ein Beurteilungsrahmen~$\Theta$. Eine Abbildung $m$
  über der Menge aller Teilmengen des Beurteilungsrahmens
  ($\{x | x \subseteq \Theta\}$) mit
  %
  \begin{enumerate}
  \item $m_\Theta(x) \geq 0$,
  \item $m_\Theta(\emptyset) = 0$ und
  \item $\sum_{x \subseteq \Theta} m_\Theta(x) = 1$
  \end{enumerate}
  %
  heißt \emph{Überzeugungsstärkeverteilung} (belief mass
  distribution, basic probability assignment). Der Wert von
  $m_\Theta(x)$ wird als \emph{Überzeugungsstärke} (belief mass,
  basic probability number) in die Behauptung $x$ bezeichnet.
  \cite{Joesang2007}, \cite{Shafer1976}
\end{definition}
\end{verbatim}

\begin{verbatim}
\begin{theorem}
  \label{th:ueberzeugungssumme}
  Die Summe aus Überzeugung, Gegenüberzeugung und Unsicherheit
  ergibt immer Eins:
  %
  \begin{align*}
    \Bel(A) + \operatorname{Dou}(A) + u(A) = 1, \qquad A \in
    \{x | x \subseteq \Theta\},\ A \neq \emptyset.
  \end{align*}
\end{theorem}
\end{verbatim}

\paragraph{Weitere Dokumentation}

Weitere Dokumentation zu den Funktionen der AMS-Pakete finden Sie in
amsldoc.pdf.




\section{Verweise}%
\label{sec:guide-verweise}


\subsection{Literaturverzeichnis und \BibTeX}%
\label{sec:literaturverzeichnis}

Mit dem Paket \emph{natbib} lassen sich Literaturverweise flexibel
gestalten. Deshalb wird es automatisch von den LDV-Dokumentenklassen
eingebunden. Es bietet beispielsweise den Befehl \verb|\citet|, um
einen Verweis zu erzeugen, bei dem der Name des Autors in den Text
integriert ist (\enquote{Müllers (2002) wählte den Ansatz \ldots}).

\DescribeMacro{\bibliographystyle}%
Damit die LDV-Klassen ein komplettes Stilpaket anbieten, stellen sie
bereits einen Bibliographiestil als Vorgabewert ein. Es heißt einfach
\emph{ldv}. Dieser Stil unterstützt folgende zusätzliche
\BibTeX-Attribute:
%
\begin{itemize}
\item \verb|isbn|
\item \verb|issn|
\item \verb|doi|
\item \verb|url|
\item \verb|language|
\end{itemize}

Sie können also eine \verb|\bibliographystyle|-Anweisung bei den
LDV-Klassen einfach weglassen, weil ein Bibliographiestil bereits
voreingestellt ist. Sollten Sie einen anderen Stil wollen, können sie
diesen natürlich ganz normal mit eben jenem Befehl auswählen.

natbib unterscheidet den Bibliographiestil vom Zitierstil. Ersterer
gestaltet das Literaturverzeichnis, zweiterer den Verweis aus dem Text
zu einem Eintrag im Literaturverzeichnis. Standardmäßig verwenden die
LDV-Klassen den Zitierstil \emph{ldv}, welcher das Autor-Jahr-Schema
(z.B. \enquote{Meyer (2002)}) verwendet.  Wollen Sie dagegen lieber Verweise
nach dem numerische Schema (z.B. \enquote{[12]}), dann wählen Sie den
Zitierstil \emph{ldvplain}. Sie können ihn mit dem Befehl
%
\begin{verbatim}
\citestyle{ldvplain}
\end{verbatim}
%
aktivieren. Sie müssen dafür nicht zu einem anderen Bibliographiestil
wechseln. Weitere Zitierstile finden Sie in der Dokumentation zum
natbib-Paket.


\paragraph{Weitere Dokumentation.}

Das Dokument natbib.pdf erläutert im Detail, wie man verschiedenste
Formen von wissenschaftlichen Literaturverweisen in \LaTeX\ umsetzen
kann. Im Unterschied zu dieser Dokumentation generiert der Befehl
\verb|\cite| in der LDV-Konfiguration immer Klammern um den Verweis.





\subsection{Verweise innerhalb des Dokuments}%
\label{sec:verw-innerh-des}

\DescribeMacro{\vref}%
Die LDV-Dokumentenklassen binden das Paket \emph{varioref} ein. Es
bietet im Wesentlichen den Befehl \verb|\vref|, der wie \verb|\ref|
benutzt wird und einen intelligenten Verweis erzeugt -- zum Beispiel
\enquote{Abbildung 4.1 auf der vorherigen Seite}. Ich kann \verb|\vref| vor
allem für Verweise auf Abbildungen, Tabellen, usw. empfehlen. Kapitel
kann man dagegen leicht durch Kolumnentitel finden, so dass für diese
der normale \LaTeX-Befehl \verb|\ref| gut geeignet ist.

\DescribeMacro{\ref}\DescribeMacro{\vref}%
Zusätzlich sind die verweisenden Befehle wie beispielsweise
\verb|\ref|, \verb|\vref| und \verb|\pageref| durch das eingebundene
Paket \emph{hyperref} automatisch anklickbare Links bei PDF- oder
HTML-Ausgabe.

\paragraph{Weitere Dokumentation.}

Das Dokument varioref.pdf beschreibt Mög\-lich\-kei\-ten,
verschiedenste Verweise innerhalb eines Dokuments zu realisieren. In
manual.pdf finden Sie darüber hinaus Informationen zu elektronischen
Verweisen mit \LaTeX.



\subsection{Externe Verweise}%
\label{sec:externe-verweise}

\DescribeMacro{\url}\DescribeMacro{\href}%
Mit dem hyperref-Paket kann man auch URLs angeben, die dann als Link
anklickbar sind. Dazu dienen die Befehle \verb|\url| und
\verb|\href|. Das Paket hyperref ist sehr umfangreich, so dass man bei
Wünschen in Sachen Verweise einmal einen Blick in die Dokumentation
werfen sollte.

\paragraph{Weitere Dokumentation.}

Details zu elektronischen Verweisen mit \LaTeX\ stehen im Dokument
manual.pdf.



\subsection{Probleme mit hyperref und varioref}

\DescribeOption{omitpackage}%
Insbesondere das hyperref-Paket verändert die \LaTeX-Umgebung
weitreichend. Es ist dadurch inkompatibel zu manchen anderen
Paketen. Deshalb können Sie hyperref und varioref bei Bedarf
abschalten. Die Klassendefinition
%
\begin{verbatim}
\documentclass[omitpackage=hyperref,omitpackage=varioref]{ldvarticle}
\end{verbatim}
%
verhindert, dass die beiden Pakete automatisch geladen werden. Das ist
auch nützlich, wenn Sie diese Pakete mit Ihren eigenen Optionen laden
wollen.




\section{Layoutanpassung}%
\label{sec:layoutanpassung}

\begin{itemize}
\item \emph{color}-Paket bereits eingebunden.

\item Vordefinierte Farben der
  Corporate Identity (wie in Broschüre benannt, in CMYK-Farben für den
  Druck):
  \begin{itemize}
  \item TUMBlau
  \item TUMBlau1
  \item TUMBlau2
  \item TUMBlau3
  \item TUMBlau4
  \item TUMBlau5
  \item TUMDunkelgrau
  \item TUMMittelgrau
  \item TUMHellgrau
  \item TUMGruen
  \item TUMOrange
  \item TUMElfenbein
  \end{itemize}

\item Basisschrift wählen mit der Klassenoption \verb|fontstyle|:
  \begin{itemize}
  \item \verb|fontstyle=sans|: Serifenlose Schrift für
    Fließtext (Voreinstellung).
  \item \verb|fontstyle=serif|: Serifenschrift für Fließtext
  \end{itemize}

\item Der \verb|\tolerance|-Wert von \TeX\ ist bei diesen
  Dokumentenklassen auf 800 voreingestellt, damit es weniger übervolle
  H-Boxen gibt bei gleichzeitig gutem Schriftbild.
\end{itemize}

\StopEventually{}

\PrintChanges

\section{Implementierung}%
\label{sec:implementierung}

\subsection{Roadmap/Todo}
Diese Roadmap hat aktuell nur noch historischen Wert und keinerlei praktische Relevanz.

\paragraph{Version 2.0}

\begin{itemize}
\item Umschlag
  \begin{itemize}
  \item Umschlagseite als PDF-Datei einbindbar.
  \item Overful hbox in Design1 (tb\_vielMeta.tex)
  \end{itemize}

\item Impressumsseite
  \begin{itemize}
  \item Impressumsseite wird standardmäßig mit \verb|\maketitle| bei
    doppelseitigem Layout generiert. Siehe Kapitel 6.3 dieser
    Dokumentation.
  \item Parse Jahr aus Datum, egal welche Sprache
  \item Lizenzen in Impressum integrieren. Lokalisiert.
  \end{itemize}

\item Underful hbox im Inhaltsverzeichnis (tb.tex)

\item Literaturverzeichnis
  \begin{itemize}
  \item Eigenen Bibliographiestil -- Fertig
  \item ldv.bst und \verb|\citestyle{plain}| in ldvguide
  dokumentieren. -- Fertig
  \end{itemize}

\item Windows-Installation neu dokumentieren und Aufbau der
  Distribution an übliche Konvention anpassen.

\item \verb|\tolerance|-Wert und sonstige Satzoptionen neu
  justieren. -- Fertig

\item Implementierung
  \begin{itemize}
  \item Vorgabewerte nicht in eigenem Kapitel sondern dort, wo sie
    genutzt werden. -- Fertig
  \item Pakete für Referenzen abschaltbar machen wegen
    Kompatibilitätsproblemen. -- Fertig
  \end{itemize}
\end{itemize}

\paragraph{Version 3.0}

\begin{itemize}
\item Auf etoolbox-Paket umstellen
\item Kopfzeile passt auf Kapitelseite nicht (Abstand). Neu machen
  mit scrheadings, so dass auch die Seitenzahl in der Kopfzeile ist.
\item Titelseite für Doktorarbeit
\item Weitere Umschlagdesigns anbieten
\item Schlüsselwörter in Titelei integrieren und \verb|makekeywords|
  für einspaltiges Layout anpassen.
\item Feld für den Herausgeber (Editor), der ediert. ?? sinnvoll
  und nötig?
\item Verhältnis Version, Auflage klären und umsetzen
\item Das Makro \verb|\and| so umsetzen, dass es mit vorausgehenden
  Leerzeichen umgehen kann und passend Kommas sowie \enquote{und} einfügt.
\item Kurzreferenz schreiben.
\item Formeldarstellung im Zusammenhang mit helvet-Option
  verbessern. Soll Helvetica dann die Standardschrift werden oder
  doch die serif-Option? sfmath.sty integrieren
\end{itemize}

\paragraph{Version 4.0}

\begin{itemize}
\item Integration mit \KOMAScript Version 3
\item Diplomarbeitsanleitung schreiben.
\item Literatur-Typen www und media (mit IEEE-Stil vergleichen).
\item Farbiges Layout
\item Gesamtlayout überarbeiten und fixieren
\item Verbesserter Blindtext
\item Vorlagen für LyX
\item Dynamisches Layout, welches auch für DIN A3 und A5, sowie
  (angelsächsische) Zwischengrößen funktioniert.
\end{itemize}


\subsection*{Anforderungen}

Die Testdokumente nutzen das Paket \emph{blindtext},
welches bei vielen Distributionen nachinstalliert werden muss.





\DocInput{ldvcommon.dtx}

\clearpage
%\enlargethispage{-\baselineskip}


\PrintIndex

\Finale
\end{document}


%%% Local Variables:
%%% TeX-PDF-mode: t
%%% End:
